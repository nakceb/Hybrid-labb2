\documentclass[10pt,a4paper]{article}
\usepackage[utf8]{inputenc}
\usepackage{amsmath}
\usepackage{amsfonts}
\usepackage{amssymb}
\usepackage{graphicx}
\author{FILIP STENBECK	930702-4530 \\ ALEXANDER RAMM 931005-8418}
\date{Febuary 2016}
\title{Homework 2}
\begin{document}
\maketitle
\clearpage
\section*{Part 1:}
\section*{Part 2:}
\subsection*{1. " Calculate analytically the closed-loop equations of the system."}
The following network control system is given. 
\begin{equation}
\dot{x}=Ax+Bu
\end{equation}\\ Where the matrices $A$, $B$ are set to be $0$, $I$ since the system is a simple integrator. The input $u$ is set to be. \\
\begin{equation*}
u=-Kx
\end{equation*}\\
In the network two delays are introduced. One before and one after the controller. This results in that the input of the system is delayed $\tau_{sc}+\tau_{ea}$ seconds. Since $\tau_{sc}+\tau_{ea}<h$ the input signal can be written.\\
\begin{equation}\label{knep1}
u=
\begin{cases}
u_0=-Kx(kh-h)&kh<t<kh+\tau_{sc}+\tau_{ea}\\
u_1=-Kx(kh)&kh+\tau_{sc}+\tau_{ea}<t<kh+h
\end{cases}
\end{equation}
With this we can formulate the closed loop equation.
\begin{equation*}
x(kh+h)=\mathrm{e}^{Ah}+\int_{kh}^{kh+h} \mathrm{e}^{AS}\,\mathrm{d}S Bu
\end{equation*}
\begin{equation}\label{knep2}
=\mathrm{e}^{Ah}x(kh)+\int_{kh}^{kh+\tau_{sc}+\tau_{ea}} \mathrm{e}^{AS}\,\mathrm{d}S Bu_0+\int_{kh+\tau_{sc}+\tau_{ea}}^{kh+h} \mathrm{e}^{AS}\,\mathrm{d}S Bu_1
\end{equation}
Setting the time delay $\tau=\tau_{sc}+\tau_{ea}$ and combining the equations \ref{knep1} with \ref{knep2} we result in the following closed loop equation.
\begin{equation*}
x(kh+h)=\begin{bmatrix}
1+K(\tau-h)&0\\
0&-K\tau
\end{bmatrix}
\begin{bmatrix}
x(kh)\\
x(kh-1)
\end{bmatrix}
\end{equation*}
\subsection*{2}
Since the continuous plant is a simple integrator $G(s)=\frac{1}{s}$ we get that the zero-order hold of the continuous plant will have the following pulse-transfer function.
\begin{equation*}
H(z)=\frac{h}{z-1}
\end{equation*}
With the controller as $C(z)=-K$ we will have a stable system if.
\begin{equation*}
|\frac{G(z)C(z)}{1+G(z)C(z)}|<\frac{1}{\tau z},	z\in R
\end{equation*}
This can be written as.
\begin{equation*}
hK(1-z\tau)<z-1,	z\in R
\end{equation*}





\end{document}